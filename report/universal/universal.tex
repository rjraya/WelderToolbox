\subsection{Universal algebra}

\subsubsection{General notions}

A way to formalize the notion of computation is to introduce multi-typed algebraic structures:

\begin{definition}[Signature and algebra]
A \textbf{signature} is given by:

\begin{enumerate}
\item A set $S \neq \emptyset$ of types or sorts.
\item An indexed family in $S^{\star} \times S$: $$\{\Sigma_{w,s}:w \in S^{\star}, s \in S\}$$ where $c \in \Sigma_{\epsilon,s}$ is a constant of type $s$ and $\sigma \in \Sigma_{w = s(1) \ldots s(n),s}$ is an operation of type $(w,s)$ and arity $n$. 
\end{enumerate}

Normally, one denotes a signature as $\Sigma = (S,\{\Sigma_{w,s}:w \in S^{\star}, s \in S\})$.

A \textbf{$\Sigma$-algebra} consists of:

\begin{enumerate}
\item A family $\{A_s:s \in S \}$ of data types.
\item A family $\{\Sigma^A_{w,s}:w \in S^{\star}, s \in S\}$ such that:

\begin{enumerate}
\item $\forall s \in S.\Sigma_{\epsilon,s}^A = \{c_A:c \in \Sigma_{\epsilon,s} \}$ where $c_A \in A_s$ is a constant of type $s$ that instantiates the constant $c \in \Sigma_{\epsilon,s}$ for this algebra.

\item $\forall w \in S^+,s \in S.\Sigma_{w,s}^A =\{\sigma_A:\sigma \in \Sigma_{w,s} \}$ where $\sigma_A \in \Sigma_{w = s(1) \ldots s(n),s}^A$ is a function: $$\sigma_A:A_{s(1)} \times \ldots \times A_{s(n)} \mapsto A_s$$ that interprets the function symbol $\sigma$ for this algebra.
\end{enumerate}
\end{enumerate}

In other words, a $\Sigma$-algebra is an instantiation of the signature $\Sigma$ for particular data-types. It is denoted $(\{A_s:s \in S \},\{\Sigma^A_{w,s}:w \in S^{\star}, s \in S\})$.
\end{definition}

\begin{example}
In the following examples, triples $(a;b;c)$ denote the set of data-types, constants and operations in this order. Note that the signature is implicit in most of the examples. 

\begin{enumerate}
\item Let $B = \{ \top,\bot \}$ be the set of truth values. Then $(B;\top,\bot; \lnot , \land , \lor , \oplus , \implies , \iff )$ is an algebra.

\item Let $\mathbb{N}$ denote the set of natural numbers and $Succ$ be the successor function $Succ(n) = n+1$. Then $(\mathbb{N};0;Succ)$ is an algebra.

\item Algebraic structures such as rings or fields are clearly examples of algebras.

\item The most important example for us is the algebra of terms. In this case we explicitly name the underlying signature. 

The signature $\Sigma$ will be given by a unique  data type $S = \{s\}$ a set of constants denoted $F_0$ and a family of functions denoted with $F = \cup_{n \in \mathbb{N}} F_n$ where $F_n$ gathers all the functions with arity $n$. 

The instantiation data-type is the set of terms: $$T(\Sigma, X) = F_0 \cup X \cup \{f(t_1,\ldots,t_n):t_i \in T(\Sigma,X) \land f \in F_n \land n \in \mathbb{N} \}$$ Here $X$ is a set of variable symbols of type $s$ such that $X \cap F_0 = \emptyset$. 

Finally, the algebra of terms is given by $T(\Sigma,X)$ as data-type, $F_0$ as a set of constants and a set of operations built from the signature: for each $f \in F_n$ one gives $F:T(\Sigma,X)^n \to T(\Sigma,X)$ such that $(t_1,\ldots,t_n) \mapsto f(t_1,\ldots,t_n)$.

The algebra of terms is usually denoted by $T(\Sigma,X)$ (same notation as the underlying datatype).
\end{enumerate}
\end{example}

As usual in mathematics, it is interesting to introduce the notion of substructure and homomorphism between structures:

\begin{definition}[$\Sigma$-sub-algebra]
Let $A,B$ be $\Sigma$-algebras indexed in $S$. $B$ is a sub-algebra of $A$, which we write, $B \le A$ if:

\begin{enumerate}
\item $\forall s \in S.B_s \subseteq A_s$
\item $\forall s \in S,c \in \Sigma_{\epsilon,s}.c_B = c_A$. 
\item $\forall s \in S,w \in S^{+},\sigma \in \Sigma_{w,s},(b_1,\ldots,b_n) \in B^{w}.\sigma_B(b_1,\ldots,b_n) = \sigma_A(b_1,\ldots,b_n)$
\end{enumerate}
\end{definition}

\begin{definition}[Homomorphism of algebras]
Let $A,B$ be two $\Sigma$-algebras indexed in $S$. 

A $\Sigma$-homomorphism $\phi:A \to B$ is an $S$-indexed family of mappings $\phi = \langle \phi_s:A_s \to B_s:s \in S \rangle$ such that:

\begin{enumerate}
\item $\forall s \in S,c \in \Sigma_{\epsilon,s}.c_B = \phi_s(c_A)$
\item $\forall w = s(1) \ldots s(n) \in S^+,s \in S,\sigma \in \Sigma_{w,s},(a_1,\ldots,a_n) \in A^w.\phi_s(\sigma_A(a_1,\ldots,a_n)) = \sigma_B(\phi_{s(1)}(a_1),\ldots,\phi_{s(n)}(a_n))$
\end{enumerate}
\end{definition}

Of course, the notation is greatly simplified if we consider only one underlying data-type $S = \{s\}$.

Our goal now is to show how the algebra of terms can be formally studied to make useful deductions.

\subsubsection{Syntactic characterization of $\stackrel{*}{\leftrightarrow}$}

\begin{definition}[Positions and size of a term]
Given terms $s \in T(\Sigma,X)$ its \textbf{set of positions} is defined inductively as:

\begin{enumerate}
\item $s = x \in X \implies Pos(s) = \{\epsilon\}$
\item $s = f(s_1,\cdots,s_n) \implies Pos(s) = \{\epsilon\} \bigcup \cup_{i=1}^{n} \{ip:p \in Pos(s_i)\}$
\end{enumerate}

We call the position $\epsilon$ root position and the function or variable symbol at this position is called the root symbol. 

The \textbf{size} of $s \in T(\Sigma,X)$ is $|s| = |Pos(s)|$. 
\end{definition}

\begin{definition}[Order on positions]
The prefix order is a partial order on the set of positions: 

$p \leq q \iff \exists p'. pp' = q$

With respect to $\leq$, we say $p,q$ are:

\begin{enumerate}
\item parallel, $p||q$, if they are incomparable.
\item $p$ is above $q$, if $p \leq q$.
\item $p$ is strictly above $q$, if $p < q$.   
\end{enumerate} 
\end{definition}


\begin{definition}[Term operations]
Given $p \in Pos(s)$ the \textbf{sub-term of s at position p}, $s|_p$, is defined by induction on the length of p:

\begin{enumerate}
\item $s|_\epsilon = s$
\item $f(s_1,\cdots,s_n)|_{iq} = s_i|_q$. 
\end{enumerate}

The term obtained from $s$ by \textbf{replacing the sub-term at position p by t}, $s[t]_p$ is:

\begin{enumerate}
\item $s[t]_\epsilon = t$
\item $f(s_1,\cdots,s_n)[t]_{iq} = f(s_1,\cdots,s_i[t]_q,\cdots,s_n)$
\end{enumerate}

Finally, $Var(s) = \{x \in X:\exists p \in Pos(s).s|_p = x\}$ denotes the \textbf{set of variables occurring in s} and the corresponding positions are called variable positions.
\end{definition}

\begin{definition}[Ground terms]
$s \in T(\Sigma,X)$ is a \textbf{ground} term if $Var(s) = \emptyset$. $T(\Sigma)$ denotes the set of all ground terms over $\Sigma$.
\end{definition}

\begin{definition}[Substitutions]
Suppose that $V$ is a countably infinite set of variables. 

A \textbf{$T(\Sigma,V)$-substitution} is a function $\sigma:V \rightarrow T(\Sigma,V)$ such that $\sigma(x) \neq x$ for only finitely many variables. Associated with a substitution we have the following notions:

\begin{enumerate}
\item The \textbf{domain} of $\sigma$ is $Dom(\sigma) = \{x \in V: \sigma(x) \neq x\}$.
\item The \textbf{range} of $\sigma$ is the set $Ran(\sigma) = \{\sigma(x):x \in Dom(\sigma)\}$.
\item The \textbf{variable range} of $\sigma$ is $VRan(\sigma) = \cup_{x \in Dom(\sigma)} Var(\sigma(x))$.
\end{enumerate}

$Sub(T(\Sigma,V))$ denotes the set of all $T(\Sigma,V)$-substitutions.

We \textbf{extend} the application of a substitution to terms as follows:

Given $\sigma \in Sub$, define $\hat\sigma: T(\Sigma,V) \to T(\Sigma,V)$ such that $
\hat\sigma(x)=
\begin{cases}
 \sigma(x)&\text{if}\, x \in V\\
 f(\hat\sigma(s_1),\cdots,\hat\sigma(s_n))&\text{if}\, s = f(s_1,\cdots,s_n)
\end{cases}$

Then one can define the composition of $\sigma$ and $\tau$ as $\sigma\tau(x) = \hat\sigma(\tau(x))$. This has the following properties:

\begin{enumerate}
\item $\sigma\tau$ is again a substitution.
\item The composition of substitutions is associative.
\item $\hat{\sigma\tau} = \hat\sigma \hat\tau$
\end{enumerate}

An instance of term $s$ is a term $t$ such that  $\exists \sigma$ substitution such that $\sigma(s) = t$.  We note it as $t \gtrsim s$ and $t > s$ if the opposite relation does not hold. 
\end{definition}

\begin{definition}[Identities and reduction relation]
An \textbf{$\Sigma$-identity} is a pair $(s,t) \in T(\Sigma,V) \times T(\Sigma,V)$. We write $s \approx t$ and call $s$ left-hand side and $t$ right-hand side.  

Let $E$ be a set of $\Sigma$-identities. The \textbf{reduction relation} $\to_E \subseteq T(\Sigma,V) \times T(\Sigma,V)$ is defined as $$s \to_E t \iff \exists (l,r) \in E, p \in Pos(s), \sigma \in Sub. s|_p = \sigma(l) \land t = s[\sigma(r)]_p$$
\end{definition}

This relation is quite intuitive. We take the equation $(l,r)$ as a template, whose left part matches up to a transformation of a sub-term of the source term $s$. Then, we apply the equivalence $(l,r)$ and substitute the right term modified properly. This should give us all the derived terms. 

\begin{definition}[Properties of binary relations on $T(\Sigma,V)$]
Let $\equiv$ be a binary relation on $T(\Sigma,V)$. $\equiv$ is:

\begin{enumerate}
\item \textbf{closed under substitutions:} 

$\forall s,t \in T(\Sigma,V), \sigma \in Sub. s \equiv t \implies \sigma(s) \equiv \sigma(t) $
\item \textbf{closed under $\Sigma$-operations:}

$\forall f \in \Sigma^{ar(f)},s_i,t_j \in T(\Sigma,V). (\forall i. s_i \equiv t_i) \implies f(s_1,\ldots,s_{ar(f)}) \equiv f(t_1,\ldots,t_{ar(f)}) $
\item \textbf{compatible with $\Sigma$-operations:} 

$\forall f \in \Sigma^{arg(f)}, s_j,s,t \in T(\Sigma,V). s \equiv t \implies f(s_1,\ldots,s_{i-1},s,s_{i+1},\ldots,s_{ar(f)}) \equiv f(s_1,\ldots,s_{i-1},t,s_{i+1},\ldots,s_{ar(f)})$
\item \textbf{compatible with $\Sigma$-contexts:}

$\forall t \in T(\Sigma,V),p \in Pos(t). s \equiv s' \implies t[s]_p \equiv t[s']_p$
\end{enumerate}
\end{definition}

\begin{theorem}[Characterization of $\stackrel{*}{\leftrightarrow}_E$]\label{3.1.12}
1. Let E be a set of $\Sigma$-identities. The reduction relation $\to_E$ is closed under substitutions and compatible with $\Sigma$-operations. \\
2. Given a binary relation $\equiv$ on $T(\Sigma,V)$:
\begin{itemize}
\item $\equiv$ is compatible with $\Sigma$-operations $\iff$ it is compatible with $\Sigma$-contexts.
\item If $\equiv$ is reflexive and transitive then it is compatible with $\Sigma$-operations $\iff$ it is closed under $\Sigma$-operations.
\end{itemize}

3. Let E be a set of $\Sigma$-identities. The relation $\stackrel{*}{\leftrightarrow}_E$ is the smallest equivalence relation on $T(\Sigma,V)$ that contains E and is closed under substitutions and $\Sigma$-operations.
\end{theorem}
\begin{proof}
1. Follows from definition.
2. Definition and induction.
3. Clearly, $\stackrel{*}{\leftrightarrow}$ contains E. By definition, $\stackrel{*}{\leftrightarrow}$ is an equivalence relation. Using parts 1,2 one shows that $\stackrel{*}{\leftrightarrow}$ is closed under substitutions and $\Sigma$-operations.

Finally, one assumes that $\equiv$ is another equivalence relation containing E and closed under substitutions and $\Sigma$-operations. Then one shows that $s \to_E t \implies s \equiv t$. Since by definition, $\stackrel{*}{\leftrightarrow}$ is the smallest equivalence relation containing $\to_E$, we deduce that $\stackrel{*}{\leftrightarrow} \subseteq \equiv$.
\end{proof}

The last property, tells us that $\stackrel{*}{\leftrightarrow}_E$ is obtained from E, closing it under reflexivity, symmetry, transitivity, substitutions and $\Sigma$-operations. This process can be described with inferences rules and leads to \textbf{equational logic}:

\begin{table}[H]
\centering
\begin{tabular}{|| c | c | c | c ||}
\hline
\hline \infer{E \vdash s \approx t}{ (s \approx t) \in E } \\
\hline \infer{E \vdash t \approx t}{\vspace{1.5mm}}  \\
\hline \infer{E \vdash t \approx s}{E \vdash s \approx t} \\
\hline \infer{E \vdash s \approx u}{E \vdash s \approx t & t \approx u} \\
\hline \infer{E \vdash \sigma(s) \approx \sigma(t)}{E \vdash s \approx t} \\
\hline \infer{E \vdash f(s_1,\ldots,s_n) \approx f(t_1,\ldots,t_n)}{E \vdash s_1 \approx t_1 & \ldots & s_n \approx t_n} \\
\hline
\end{tabular}
\caption{Inference rules of equational logic}
\label{table:inf1}
\end{table}

We note $E \vdash s \approx t$ to specify that $s \approx t$ can be obtained from $E$ applying the above rules. We read it as "$s \approx t$ is a syntactic consequence of E". The first rule receives the special name of \textbf{assumption rule}. One can also build \textbf{proof trees} by composing inference rules in the usual manner. There are however differences to note between both approaches:

1. The rewriting approach given by the computation of $\stackrel{*}{\leftrightarrow}$ allows the replacement of a sub-term at an arbitrary position in a single step (choosing $l \approx r,\sigma,p$). The inference rule approach needs to simulate this by many small steps of closure under single operations. 

2. Closure under operations in the inference rule approach allows the simultaneous replacement in each argument of an operation. The rewriting approach needs to simulate this by a number of replacements steps in sequence. 

\subsubsection{Semantic characterization of $\stackrel{*}{\leftrightarrow}$}

\begin{definition}[Models for a set of identities]
Let $\mathcal{A}$ be a $\Sigma$-algebra  and $s \approx t$ a $\Sigma$-identity:

$s \approx t$ \textbf{holds} in $\mathcal{A} \iff \forall \phi:T(\Sigma,V) \to A$ homomorphism.$\phi(s) = \phi(t)$.

In that case, we denote it as $\mathcal{A} \models s \approx t$.

Let E be a set of $\Sigma$-identities. 

$\mathcal{A}$ is a \textbf{model} of E $\iff \forall (s \approx t) \in E. \mathcal{A} \models s \approx t$. 

In that case, we denote it as $\mathcal{A} \models E$ and $\nu(E)$ the class of all models or  \textbf{$\Sigma$-variety} defined by E.
\end{definition}

Here we stress the use of word "identity" for a pair of term $s \approx t$ to express that this equality is assumed to hold in an algebra. The word "equation" is used to express that the equality must be solved in an algebra. This reduces to a universally or existentially quantified goal. 

\begin{definition}[Equational theory]
$s \approx t$ is a \textbf{semantic consequence} of E $\iff \forall \mathcal{A} \in \nu(E).\mathcal{A} \models s \approx t$.

In that case, we denote it as $E \models s \approx t$.

The relation $\approx_E = \{(s,t) \in T(\Sigma,V) \times T(\Sigma,V):E \models s \approx t\}$ is the \textbf{equational theory} induced by E. 
\end{definition}

\begin{theorem}[Birkhoff's theorem]
Let E be a set of identities. 

The syntactic consequence relation $\stackrel{*}{\leftrightarrow}$ coincides with the semantic consequence relation $\approx_E$. 

Alternatively, we say that $\vdash$ and $\models$ coincide. 
\end{theorem}

The meaning of these theorem is that the syntactic procedure always leads to valid equations and that any valid equations can be obtained through the syntactic procedure. 



